% Options for packages loaded elsewhere
\PassOptionsToPackage{unicode}{hyperref}
\PassOptionsToPackage{hyphens}{url}
%
\documentclass[
]{book}
\usepackage{lmodern}
\usepackage{amssymb,amsmath}
\usepackage{ifxetex,ifluatex}
\ifnum 0\ifxetex 1\fi\ifluatex 1\fi=0 % if pdftex
  \usepackage[T1]{fontenc}
  \usepackage[utf8]{inputenc}
  \usepackage{textcomp} % provide euro and other symbols
\else % if luatex or xetex
  \usepackage{unicode-math}
  \defaultfontfeatures{Scale=MatchLowercase}
  \defaultfontfeatures[\rmfamily]{Ligatures=TeX,Scale=1}
\fi
% Use upquote if available, for straight quotes in verbatim environments
\IfFileExists{upquote.sty}{\usepackage{upquote}}{}
\IfFileExists{microtype.sty}{% use microtype if available
  \usepackage[]{microtype}
  \UseMicrotypeSet[protrusion]{basicmath} % disable protrusion for tt fonts
}{}
\makeatletter
\@ifundefined{KOMAClassName}{% if non-KOMA class
  \IfFileExists{parskip.sty}{%
    \usepackage{parskip}
  }{% else
    \setlength{\parindent}{0pt}
    \setlength{\parskip}{6pt plus 2pt minus 1pt}}
}{% if KOMA class
  \KOMAoptions{parskip=half}}
\makeatother
\usepackage{xcolor}
\IfFileExists{xurl.sty}{\usepackage{xurl}}{} % add URL line breaks if available
\IfFileExists{bookmark.sty}{\usepackage{bookmark}}{\usepackage{hyperref}}
\hypersetup{
  pdftitle={The pmtables Book},
  pdfauthor={Metrum Research Group},
  hidelinks,
  pdfcreator={LaTeX via pandoc}}
\urlstyle{same} % disable monospaced font for URLs
\usepackage[margin=1in]{geometry}
\usepackage{color}
\usepackage{fancyvrb}
\newcommand{\VerbBar}{|}
\newcommand{\VERB}{\Verb[commandchars=\\\{\}]}
\DefineVerbatimEnvironment{Highlighting}{Verbatim}{commandchars=\\\{\}}
% Add ',fontsize=\small' for more characters per line
\usepackage{framed}
\definecolor{shadecolor}{RGB}{248,248,248}
\newenvironment{Shaded}{\begin{snugshade}}{\end{snugshade}}
\newcommand{\AlertTok}[1]{\textcolor[rgb]{0.94,0.16,0.16}{#1}}
\newcommand{\AnnotationTok}[1]{\textcolor[rgb]{0.56,0.35,0.01}{\textbf{\textit{#1}}}}
\newcommand{\AttributeTok}[1]{\textcolor[rgb]{0.77,0.63,0.00}{#1}}
\newcommand{\BaseNTok}[1]{\textcolor[rgb]{0.00,0.00,0.81}{#1}}
\newcommand{\BuiltInTok}[1]{#1}
\newcommand{\CharTok}[1]{\textcolor[rgb]{0.31,0.60,0.02}{#1}}
\newcommand{\CommentTok}[1]{\textcolor[rgb]{0.56,0.35,0.01}{\textit{#1}}}
\newcommand{\CommentVarTok}[1]{\textcolor[rgb]{0.56,0.35,0.01}{\textbf{\textit{#1}}}}
\newcommand{\ConstantTok}[1]{\textcolor[rgb]{0.00,0.00,0.00}{#1}}
\newcommand{\ControlFlowTok}[1]{\textcolor[rgb]{0.13,0.29,0.53}{\textbf{#1}}}
\newcommand{\DataTypeTok}[1]{\textcolor[rgb]{0.13,0.29,0.53}{#1}}
\newcommand{\DecValTok}[1]{\textcolor[rgb]{0.00,0.00,0.81}{#1}}
\newcommand{\DocumentationTok}[1]{\textcolor[rgb]{0.56,0.35,0.01}{\textbf{\textit{#1}}}}
\newcommand{\ErrorTok}[1]{\textcolor[rgb]{0.64,0.00,0.00}{\textbf{#1}}}
\newcommand{\ExtensionTok}[1]{#1}
\newcommand{\FloatTok}[1]{\textcolor[rgb]{0.00,0.00,0.81}{#1}}
\newcommand{\FunctionTok}[1]{\textcolor[rgb]{0.00,0.00,0.00}{#1}}
\newcommand{\ImportTok}[1]{#1}
\newcommand{\InformationTok}[1]{\textcolor[rgb]{0.56,0.35,0.01}{\textbf{\textit{#1}}}}
\newcommand{\KeywordTok}[1]{\textcolor[rgb]{0.13,0.29,0.53}{\textbf{#1}}}
\newcommand{\NormalTok}[1]{#1}
\newcommand{\OperatorTok}[1]{\textcolor[rgb]{0.81,0.36,0.00}{\textbf{#1}}}
\newcommand{\OtherTok}[1]{\textcolor[rgb]{0.56,0.35,0.01}{#1}}
\newcommand{\PreprocessorTok}[1]{\textcolor[rgb]{0.56,0.35,0.01}{\textit{#1}}}
\newcommand{\RegionMarkerTok}[1]{#1}
\newcommand{\SpecialCharTok}[1]{\textcolor[rgb]{0.00,0.00,0.00}{#1}}
\newcommand{\SpecialStringTok}[1]{\textcolor[rgb]{0.31,0.60,0.02}{#1}}
\newcommand{\StringTok}[1]{\textcolor[rgb]{0.31,0.60,0.02}{#1}}
\newcommand{\VariableTok}[1]{\textcolor[rgb]{0.00,0.00,0.00}{#1}}
\newcommand{\VerbatimStringTok}[1]{\textcolor[rgb]{0.31,0.60,0.02}{#1}}
\newcommand{\WarningTok}[1]{\textcolor[rgb]{0.56,0.35,0.01}{\textbf{\textit{#1}}}}
\usepackage{longtable,booktabs}
% Correct order of tables after \paragraph or \subparagraph
\usepackage{etoolbox}
\makeatletter
\patchcmd\longtable{\par}{\if@noskipsec\mbox{}\fi\par}{}{}
\makeatother
% Allow footnotes in longtable head/foot
\IfFileExists{footnotehyper.sty}{\usepackage{footnotehyper}}{\usepackage{footnote}}
\makesavenoteenv{longtable}
\usepackage{graphicx,grffile}
\makeatletter
\def\maxwidth{\ifdim\Gin@nat@width>\linewidth\linewidth\else\Gin@nat@width\fi}
\def\maxheight{\ifdim\Gin@nat@height>\textheight\textheight\else\Gin@nat@height\fi}
\makeatother
% Scale images if necessary, so that they will not overflow the page
% margins by default, and it is still possible to overwrite the defaults
% using explicit options in \includegraphics[width, height, ...]{}
\setkeys{Gin}{width=\maxwidth,height=\maxheight,keepaspectratio}
% Set default figure placement to htbp
\makeatletter
\def\fps@figure{htbp}
\makeatother
\setlength{\emergencystretch}{3em} % prevent overfull lines
\providecommand{\tightlist}{%
  \setlength{\itemsep}{0pt}\setlength{\parskip}{0pt}}
\setcounter{secnumdepth}{5}
\usepackage{booktabs}
\usepackage{amsthm}
\usepackage[utopia]{mathdesign}
\makeatletter
\def\thm@space@setup{%
  \thm@preskip=8pt plus 2pt minus 4pt
  \thm@postskip=\thm@preskip
}
\makeatother
\usepackage{float}
\usepackage{booktabs}
\usepackage{longtable}
\usepackage{threeparttable}
\usepackage{pdflscape}
\usepackage{array}
\usepackage{caption}
\usepackage[]{natbib}
\bibliographystyle{apalike}

\title{The pmtables Book}
\author{Metrum Research Group}
\date{2021-01-07}

\begin{document}
\maketitle

{
\setcounter{tocdepth}{1}
\tableofcontents
}
\hypertarget{about-this-book}{%
\chapter{About this book}\label{about-this-book}}

This is a \emph{sample} book written in \textbf{Markdown}. You can use anything that Pandoc's Markdown supports, e.g., a math equation \(a^2 + b^2 = c^2\).

The \textbf{bookdown} package can be installed from CRAN or Github:

Remember each Rmd file contains one and only one chapter, and a chapter is defined by the first-level heading \texttt{\#}.

To compile this example to PDF, you need XeLaTeX. You are recommended to install TinyTeX (which includes XeLaTeX): \url{https://yihui.name/tinytex/}.

\hypertarget{intro}{%
\chapter{Introduction}\label{intro}}

You can label chapter and section titles using \texttt{\{\#label\}} after them, e.g., we can reference Chapter \ref{intro}. If you do not manually label them, there will be automatic labels anyway, e.g., Chapter \ref{methods}.

\begin{Shaded}
\begin{Highlighting}[]
\KeywordTok{library}\NormalTok{(tidyverse)}
\KeywordTok{library}\NormalTok{(pmtables)}

\KeywordTok{stdata}\NormalTok{() }\OperatorTok\StringTok{ }\KeywordTok{stable}\NormalTok{() }\OperatorTok\StringTok{ }\KeywordTok{st_asis}\NormalTok{()}
\end{Highlighting}
\end{Shaded}

\begin{table}[H]
\centering
\setlength{\tabcolsep}{5pt} 
\begin{threeparttable}
\renewcommand{\arraystretch}{1.3}
\begin{tabular}[h]{lllllllll}
\hline
STUDY & DOSE & FORM & N & WT & CRCL & AGE & ALB & SCR \\
\hline
12-DEMO-001 & 100 mg & tablet & 80 & 71.4 & 104 & 33.7 & 4.20 & 1.06 \\
12-DEMO-001 & 150 mg & capsule & 16 & 89.4 & 122 & 24.4 & 4.63 & 1.12 \\
12-DEMO-001 & 150 mg & tablet & 48 & 81.7 & 104 & 34.4 & 3.83 & 0.910 \\
12-DEMO-001 & 150 mg & troche & 16 & 94.0 & 93.2 & 27.4 & 4.94 & 1.25 \\
12-DEMO-001 & 200 mg & tablet & 64 & 67.9 & 100 & 27.5 & 4.25 & 1.10 \\
12-DEMO-001 & 200 mg & troche & 16 & 76.6 & 99.2 & 22.8 & 4.54 & 1.15 \\
12-DEMO-002 & 100 mg & capsule & 36 & 61.3 & 113 & 38.3 & 4.04 & 1.28 \\
12-DEMO-002 & 100 mg & tablet & 324 & 77.6 & 106 & 29.9 & 4.31 & 0.981 \\
12-DEMO-002 & 50 mg & capsule & 36 & 74.1 & 112 & 37.1 & 4.44 & 0.900 \\
12-DEMO-002 & 50 mg & tablet & 324 & 71.2 & 106 & 34.1 & 4.63 & 0.868 \\
12-DEMO-002 & 75 mg & capsule & 36 & 72.4 & 105 & 38.2 & 3.89 & 0.900 \\
12-DEMO-002 & 75 mg & tablet & 288 & 71.6 & 98.9 & 34.2 & 4.49 & 0.991 \\
12-DEMO-002 & 75 mg & troche & 36 & 73.6 & 103 & 49.2 & 4.52 & 0.930 \\
\hline
\end{tabular}
\end{threeparttable}
\end{table}

\clearpage

\begin{Shaded}
\begin{Highlighting}[]
\KeywordTok{stdata}\NormalTok{() }\OperatorTok\StringTok{ }\KeywordTok{stable}\NormalTok{(}\DataTypeTok{cols_bold =} \OtherTok{TRUE}\NormalTok{) }\OperatorTok\StringTok{ }\KeywordTok{st_asis}\NormalTok{()}
\end{Highlighting}
\end{Shaded}

\begin{table}[H]
\centering
\setlength{\tabcolsep}{5pt} 
\begin{threeparttable}
\renewcommand{\arraystretch}{1.3}
\begin{tabular}[h]{lllllllll}
\hline
\textbf{STUDY} & \textbf{DOSE} & \textbf{FORM} & \textbf{N} & \textbf{WT} & \textbf{CRCL} & \textbf{AGE} & \textbf{ALB} & \textbf{SCR} \\
\hline
12-DEMO-001 & 100 mg & tablet & 80 & 71.4 & 104 & 33.7 & 4.20 & 1.06 \\
12-DEMO-001 & 150 mg & capsule & 16 & 89.4 & 122 & 24.4 & 4.63 & 1.12 \\
12-DEMO-001 & 150 mg & tablet & 48 & 81.7 & 104 & 34.4 & 3.83 & 0.910 \\
12-DEMO-001 & 150 mg & troche & 16 & 94.0 & 93.2 & 27.4 & 4.94 & 1.25 \\
12-DEMO-001 & 200 mg & tablet & 64 & 67.9 & 100 & 27.5 & 4.25 & 1.10 \\
12-DEMO-001 & 200 mg & troche & 16 & 76.6 & 99.2 & 22.8 & 4.54 & 1.15 \\
12-DEMO-002 & 100 mg & capsule & 36 & 61.3 & 113 & 38.3 & 4.04 & 1.28 \\
12-DEMO-002 & 100 mg & tablet & 324 & 77.6 & 106 & 29.9 & 4.31 & 0.981 \\
12-DEMO-002 & 50 mg & capsule & 36 & 74.1 & 112 & 37.1 & 4.44 & 0.900 \\
12-DEMO-002 & 50 mg & tablet & 324 & 71.2 & 106 & 34.1 & 4.63 & 0.868 \\
12-DEMO-002 & 75 mg & capsule & 36 & 72.4 & 105 & 38.2 & 3.89 & 0.900 \\
12-DEMO-002 & 75 mg & tablet & 288 & 71.6 & 98.9 & 34.2 & 4.49 & 0.991 \\
12-DEMO-002 & 75 mg & troche & 36 & 73.6 & 103 & 49.2 & 4.52 & 0.930 \\
\hline
\end{tabular}
\end{threeparttable}
\end{table}

\hypertarget{stable-intro}{%
\chapter{A simple table: stable}\label{stable-intro}}

\texttt{stable()} is the name of the workhorse function that is used to turn
data.frames into \texttt{TeX} tables. This chapter will introduce the \texttt{stable()} function and how to us it to create basic tables.

To illustrate usage and features of \texttt{stable()}, we will use the \texttt{stdata}
data set that comes with pmtables

\begin{Shaded}
\begin{Highlighting}[]
\NormalTok{data <-}\StringTok{ }\KeywordTok{stdata}\NormalTok{()}

\KeywordTok{head}\NormalTok{(data)}
\end{Highlighting}
\end{Shaded}

\begin{verbatim}
. # A tibble: 6 x 9
.   STUDY       DOSE   FORM    N     WT    CRCL  AGE   ALB   SCR  
.   <chr>       <chr>  <chr>   <chr> <chr> <chr> <chr> <chr> <chr>
. 1 12-DEMO-001 100 mg tablet  80    71.4  104   33.7  4.20  1.06 
. 2 12-DEMO-001 150 mg capsule 16    89.4  122   24.4  4.63  1.12 
. 3 12-DEMO-001 150 mg tablet  48    81.7  104   34.4  3.83  0.910
. 4 12-DEMO-001 150 mg troche  16    94.0  93.2  27.4  4.94  1.25 
. 5 12-DEMO-001 200 mg tablet  64    67.9  100   27.5  4.25  1.10 
. 6 12-DEMO-001 200 mg troche  16    76.6  99.2  22.8  4.54  1.15
\end{verbatim}

We can turn this data frame into a \texttt{TeX} table by passing it into \texttt{stable()}.

\begin{Shaded}
\begin{Highlighting}[]
\NormalTok{out <-}\StringTok{ }\KeywordTok{stable}\NormalTok{(data)}

\KeywordTok{head}\NormalTok{(out, }\DataTypeTok{n =} \DecValTok{10}\NormalTok{)}
\end{Highlighting}
\end{Shaded}

\begin{verbatim}
.  [1] "\\setlength{\\tabcolsep}{5pt} "                                            
.  [2] "\\begin{threeparttable}"                                                   
.  [3] "\\renewcommand{\\arraystretch}{1.3}"                                       
.  [4] "\\begin{tabular}[h]{lllllllll}"                                            
.  [5] "\\hline"                                                                   
.  [6] "STUDY & DOSE & FORM & N & WT & CRCL & AGE & ALB & SCR \\\\"                
.  [7] "\\hline"                                                                   
.  [8] "12-DEMO-001 & 100 mg & tablet & 80 & 71.4 & 104 & 33.7 & 4.20 & 1.06 \\\\" 
.  [9] "12-DEMO-001 & 150 mg & capsule & 16 & 89.4 & 122 & 24.4 & 4.63 & 1.12 \\\\"
. [10] "12-DEMO-001 & 150 mg & tablet & 48 & 81.7 & 104 & 34.4 & 3.83 & 0.910 \\\\"
\end{verbatim}

Note that we have shown the raw latex code that is generated by \texttt{stable()}. That
is to say: the output from \texttt{stable()} is a character vector of latex code
for the table. Note also that this character vector has a special class
associated with it: \texttt{stable}. That means we can write functions that recognize
this character vector as output from \texttt{stable()} and we can have those functions
process the character vector in special ways.

We can render that table in \texttt{TeX} \textbf{in the current Rmarkdown document} by
passing the text to \texttt{st\_asis()}.

\begin{Shaded}
\begin{Highlighting}[]
\NormalTok{out }\OperatorTok\StringTok{ }\KeywordTok{st_asis}\NormalTok{()}
\end{Highlighting}
\end{Shaded}

\begin{table}[H]
\centering
\setlength{\tabcolsep}{5pt} 
\begin{threeparttable}
\renewcommand{\arraystretch}{1.3}
\begin{tabular}[h]{lllllllll}
\hline
STUDY & DOSE & FORM & N & WT & CRCL & AGE & ALB & SCR \\
\hline
12-DEMO-001 & 100 mg & tablet & 80 & 71.4 & 104 & 33.7 & 4.20 & 1.06 \\
12-DEMO-001 & 150 mg & capsule & 16 & 89.4 & 122 & 24.4 & 4.63 & 1.12 \\
12-DEMO-001 & 150 mg & tablet & 48 & 81.7 & 104 & 34.4 & 3.83 & 0.910 \\
12-DEMO-001 & 150 mg & troche & 16 & 94.0 & 93.2 & 27.4 & 4.94 & 1.25 \\
12-DEMO-001 & 200 mg & tablet & 64 & 67.9 & 100 & 27.5 & 4.25 & 1.10 \\
12-DEMO-001 & 200 mg & troche & 16 & 76.6 & 99.2 & 22.8 & 4.54 & 1.15 \\
12-DEMO-002 & 100 mg & capsule & 36 & 61.3 & 113 & 38.3 & 4.04 & 1.28 \\
12-DEMO-002 & 100 mg & tablet & 324 & 77.6 & 106 & 29.9 & 4.31 & 0.981 \\
12-DEMO-002 & 50 mg & capsule & 36 & 74.1 & 112 & 37.1 & 4.44 & 0.900 \\
12-DEMO-002 & 50 mg & tablet & 324 & 71.2 & 106 & 34.1 & 4.63 & 0.868 \\
12-DEMO-002 & 75 mg & capsule & 36 & 72.4 & 105 & 38.2 & 3.89 & 0.900 \\
12-DEMO-002 & 75 mg & tablet & 288 & 71.6 & 98.9 & 34.2 & 4.49 & 0.991 \\
12-DEMO-002 & 75 mg & troche & 36 & 73.6 & 103 & 49.2 & 4.52 & 0.930 \\
\hline
\end{tabular}
\end{threeparttable}
\end{table}

Remember to only call \texttt{st\_asis()} when you are rendering tables inline in an
Rmd document. If you are sending table code to a \texttt{TeX} report, then
you will save them to a file and then include them into your report.

The remaining sections of this chapter will show you how to modify and
enhance this output in the more basic ways. We will implement separate
chapters for more complicated table manipulations.

\hypertarget{annotate-with-file-names}{%
\section{Annotate with file names}\label{annotate-with-file-names}}

pmtables can track and annotate your table with the filenames of the
R code that generated the table (\texttt{r\_file}) as well as the output file
where you write the the table \texttt{.tex} code (\texttt{output\_file)}.

To have pmtables annotate your table with these file names, pass them
in with the \texttt{r\_file} and \texttt{output\_file} arguments

\begin{Shaded}
\begin{Highlighting}[]
\NormalTok{out <-}\StringTok{ }\KeywordTok{stable}\NormalTok{(data, }\DataTypeTok{r_file =} \StringTok{"tables.R"}\NormalTok{, }\DataTypeTok{output_file =} \StringTok{"tables.tex"}\NormalTok{)}
\end{Highlighting}
\end{Shaded}

When we look at the rendered table, these names will show up as annotations
at the bottom of the table

\begin{Shaded}
\begin{Highlighting}[]
\NormalTok{out }\OperatorTok\StringTok{ }\KeywordTok{st_asis}\NormalTok{()}
\end{Highlighting}
\end{Shaded}

\begin{table}[H]
\centering
\setlength{\tabcolsep}{5pt} 
\begin{threeparttable}
\renewcommand{\arraystretch}{1.3}
\begin{tabular}[h]{lllllllll}
\hline
STUDY & DOSE & FORM & N & WT & CRCL & AGE & ALB & SCR \\
\hline
12-DEMO-001 & 100 mg & tablet & 80 & 71.4 & 104 & 33.7 & 4.20 & 1.06 \\
12-DEMO-001 & 150 mg & capsule & 16 & 89.4 & 122 & 24.4 & 4.63 & 1.12 \\
12-DEMO-001 & 150 mg & tablet & 48 & 81.7 & 104 & 34.4 & 3.83 & 0.910 \\
12-DEMO-001 & 150 mg & troche & 16 & 94.0 & 93.2 & 27.4 & 4.94 & 1.25 \\
12-DEMO-001 & 200 mg & tablet & 64 & 67.9 & 100 & 27.5 & 4.25 & 1.10 \\
12-DEMO-001 & 200 mg & troche & 16 & 76.6 & 99.2 & 22.8 & 4.54 & 1.15 \\
12-DEMO-002 & 100 mg & capsule & 36 & 61.3 & 113 & 38.3 & 4.04 & 1.28 \\
12-DEMO-002 & 100 mg & tablet & 324 & 77.6 & 106 & 29.9 & 4.31 & 0.981 \\
12-DEMO-002 & 50 mg & capsule & 36 & 74.1 & 112 & 37.1 & 4.44 & 0.900 \\
12-DEMO-002 & 50 mg & tablet & 324 & 71.2 & 106 & 34.1 & 4.63 & 0.868 \\
12-DEMO-002 & 75 mg & capsule & 36 & 72.4 & 105 & 38.2 & 3.89 & 0.900 \\
12-DEMO-002 & 75 mg & tablet & 288 & 71.6 & 98.9 & 34.2 & 4.49 & 0.991 \\
12-DEMO-002 & 75 mg & troche & 36 & 73.6 & 103 & 49.2 & 4.52 & 0.930 \\
\hline
\end{tabular}
\begin{tablenotes}[flushleft]
\item Source code: tables.R
\item Source file: tables.tex
\end{tablenotes}
\end{threeparttable}
\end{table}

\hypertarget{saving-your-stable}{%
\section{Saving your stable}\label{saving-your-stable}}

Saving your stable \textbf{can} be as easy as sending it into \texttt{writeLines()}

\begin{Shaded}
\begin{Highlighting}[]
\KeywordTok{writeLines}\NormalTok{(out, }\DataTypeTok{con =} \KeywordTok{tempfile}\NormalTok{(}\DataTypeTok{tmpdir =} \StringTok{'.'}\NormalTok{, }\DataTypeTok{fileext =} \StringTok{".tex"}\NormalTok{))}
\end{Highlighting}
\end{Shaded}

But remember that we passed in the \texttt{output\_file} argument to \texttt{stable()}
and we can use that data to save the table code to the file we named
in that argument.

Note that our \texttt{stable} object has another attribute now called \texttt{stable\_file}

\begin{Shaded}
\begin{Highlighting}[]
\KeywordTok{attributes}\NormalTok{(out)}
\end{Highlighting}
\end{Shaded}

\begin{verbatim}
. $class
. [1] "stable"
. 
. $stable_file
. [1] "tables.tex"
\end{verbatim}

This has the value that we passed in as \texttt{output\_file}. To save our table
to \texttt{stable\_file}, we call \texttt{stable\_save()}

\begin{Shaded}
\begin{Highlighting}[]
\KeywordTok{stable_save}\NormalTok{(out)}
\end{Highlighting}
\end{Shaded}

There is a \texttt{dir} argument to \texttt{stable\_save()} that we can use to to select
the directory where the file will be saved

\begin{Shaded}
\begin{Highlighting}[]
\KeywordTok{stable_save}\NormalTok{(out, }\DataTypeTok{dir =} \KeywordTok{tempdir}\NormalTok{())}
\end{Highlighting}
\end{Shaded}

And if you look at the default value for \texttt{dir} in \texttt{?stable\_save}, you'll
see that this is associated with an option called \texttt{pmtables.dir}; you
can set that option to your default output directory and your tables
will be saved there untill you change that

\begin{Shaded}
\begin{Highlighting}[]
\KeywordTok{options}\NormalTok{(}\DataTypeTok{pmtables.dir =} \KeywordTok{tempdir}\NormalTok{())}

\KeywordTok{stable_save}\NormalTok{(out)}
\end{Highlighting}
\end{Shaded}

\hypertarget{panel-intro}{%
\chapter{Group table rows with panel}\label{panel-intro}}

Paneling your table is a way to group sets of rows together into a ``panel''
with a panel header in bold. For example, we can panel a table of \texttt{mtcars} by \texttt{carb}. First, we sort
the data by carb:

\begin{Shaded}
\begin{Highlighting}[]
\NormalTok{smcars <-}\StringTok{ }\KeywordTok{arrange}\NormalTok{(smcars, carb)}
\end{Highlighting}
\end{Shaded}

Then we pass into \texttt{stable()} and name the paneling column:

\begin{Shaded}
\begin{Highlighting}[]
\KeywordTok{stable}\NormalTok{(smcars, }\DataTypeTok{panel =} \StringTok{"carb"}\NormalTok{) }\OperatorTok\StringTok{ }\KeywordTok{st_asis}\NormalTok{()}
\end{Highlighting}
\end{Shaded}

\begin{table}[H]
\centering
\setlength{\tabcolsep}{5pt} 
\begin{threeparttable}
\renewcommand{\arraystretch}{1.3}
\begin{tabular}[h]{lllllllllll}
\hline
name & mpg & cyl & disp & hp & drat & wt & qsec & vs & am & gear \\
\hline
\multicolumn{11}{l}{\textbf{1}}\\
Datsun 710 & 22.8 & 4 & 108 & 93 & 3.85 & 2.32 & 18.61 & 1 & 1 & 4 \\
Hornet 4 Drive & 21.4 & 6 & 258 & 110 & 3.08 & 3.215 & 19.44 & 1 & 0 & 3 \\
Valiant & 18.1 & 6 & 225 & 105 & 2.76 & 3.46 & 20.22 & 1 & 0 & 3 \\
Fiat 128 & 32.4 & 4 & 78.7 & 66 & 4.08 & 2.2 & 19.47 & 1 & 1 & 4 \\
Toyota Corolla & 33.9 & 4 & 71.1 & 65 & 4.22 & 1.835 & 19.9 & 1 & 1 & 4 \\
Toyota Corona & 21.5 & 4 & 120.1 & 97 & 3.7 & 2.465 & 20.01 & 1 & 0 & 3 \\
Fiat X1-9 & 27.3 & 4 & 79 & 66 & 4.08 & 1.935 & 18.9 & 1 & 1 & 4 \\
\hline \multicolumn{11}{l}{\textbf{2}}\\
Merc 240D & 24.4 & 4 & 146.7 & 62 & 3.69 & 3.19 & 20 & 1 & 0 & 4 \\
Merc 230 & 22.8 & 4 & 140.8 & 95 & 3.92 & 3.15 & 22.9 & 1 & 0 & 4 \\
Honda Civic & 30.4 & 4 & 75.7 & 52 & 4.93 & 1.615 & 18.52 & 1 & 1 & 4 \\
Porsche 914-2 & 26 & 4 & 120.3 & 91 & 4.43 & 2.14 & 16.7 & 0 & 1 & 5 \\
Lotus Europa & 30.4 & 4 & 95.1 & 113 & 3.77 & 1.513 & 16.9 & 1 & 1 & 5 \\
Volvo 142E & 21.4 & 4 & 121 & 109 & 4.11 & 2.78 & 18.6 & 1 & 1 & 4 \\
\hline \multicolumn{11}{l}{\textbf{4}}\\
Mazda RX4 & 21 & 6 & 160 & 110 & 3.9 & 2.62 & 16.46 & 0 & 1 & 4 \\
Mazda RX4 Wag & 21 & 6 & 160 & 110 & 3.9 & 2.875 & 17.02 & 0 & 1 & 4 \\
Merc 280 & 19.2 & 6 & 167.6 & 123 & 3.92 & 3.44 & 18.3 & 1 & 0 & 4 \\
Merc 280C & 17.8 & 6 & 167.6 & 123 & 3.92 & 3.44 & 18.9 & 1 & 0 & 4 \\
\hline \multicolumn{11}{l}{\textbf{6}}\\
Ferrari Dino & 19.7 & 6 & 145 & 175 & 3.62 & 2.77 & 15.5 & 0 & 1 & 5 \\
\hline
\end{tabular}
\end{threeparttable}
\end{table}

\hypertarget{group-table-columns-with-spanners}{%
\chapter{Group table columns with spanners}\label{group-table-columns-with-spanners}}

\hypertarget{tables-that-span-multiple-pages-longtable}{%
\chapter{Tables that span multiple pages: longtable}\label{tables-that-span-multiple-pages-longtable}}

\hypertarget{the-pipe-interface}{%
\chapter{The pipe interface}\label{the-pipe-interface}}

  \bibliography{book.bib,packages.bib}

\end{document}
